%% Please fill in your name and collaboration statement here.
\newcommand{\studentName}{**Takehiro Matsuzawa**}
\newcommand{\collaborationStatement}{**FILL IN YOUR COLLABORATION STATEMENT HERE \\ (See the syllabus for information)**}


%%%%%%%%%%%%%%%%%%%%%%%%%%%%%%%%%%%%%%%%%%%%%%%
\documentclass[solution, letterpaper]{cs121}
\usepackage{enumerate}
\begin{document}
\header{1}{Tuesday September 22, 2015 at 11:59pm}
\textit{Note: Finite automata (FA) drawings may be done by hand or using an online drawing tool.}
%%%%%%%%%%%%%%%%%%%%%%%%%%%%%%%%%%%%%%%%%%%%%%%
\PART{Sam and Serena}
%%%%%%%%%%%%%%%%%%%%%%%%%%%%%%%%%%%%%%%%%%%%%%%
\problem{2+2+1}{3 lines}


\subproblem Describe informally $L_1$, the language accepted by the NFA on the left.\\
\subproblem Describe informally $L_2$, the language accepted by the NFA on the right.\\
\subproblem Write down $L_1 \cap L_2$.
\\\\
$\frac{100*48}{82}=58.54$
\\\\\\\noindent {\bf Solution.}
\\\noindent 
{(A) } $\vareps \cup (01)^* \cup (10)^*$  
\\\noindent 
{(B) } $\vareps \cup (010)^* \cup (101)^*$  

$ \{\ { \text{ w} | \text{w is a empty string or }  } \} $
{(C) } Any string whose length is divisible by 6 and  it has 1 and 0 next to each other or empty string
$\vareps$ because no  

\pagebreak

\problem{3+6+(2)}{1 page}
Let $S_n = \{1, 2, 3, ..., n\}$. We can represent subsets of $S_n$ using strings of length $n$ by setting the $i^{th}$ character to 1 if element $i$ is in the subset, and 0 if it is not. For example, for $n = 3$, the subset $\{3\}$ would be represented by the string $001$, and the subset $\{2, 3\}$ would be represented by the string $011$. Let $f_n : \{0, 1\}^n \to P(S_n) $ be a function that produces the subset represented by a string, e.g. $f_3(001) = \{3\}$ and $f_3(011) = \{2, 3\}$.

\subproblem Is $f_n$ injective for all $n$? Surjective? Bijective? Informally explain each of your answers.

\subproblem In this question, we will use the machinery of DFAs to ``compute'' a function. We will label accept states with values, and the state that the DFA halts at after processing an input string will correspond to the output of the function it computes. Your task is to draw a DFA that computes the function $f_3$. \textit{Notes: } 
\begin{itemize}
\item If the DFA is given a string of length greater than 3, it should enter a labeled ``error'' non-accept state. 
\item If the DFA is given a string of length 3, it should halt in an accept state that corresponds to the correct subset of $S_3$.  Thus you should have $2^3 = 8$ accept states. 
\item No additional explanation or description of the DFA is required beyond the drawing. 
\item Label the accept states in your drawing. An example of a labelled accept state that should appear in your drawing is below:\\

\end{itemize}
\subproblem (Challenge!! Not required; worth up to 2 extra credit points.) How many states are required for a DFA that computes $f_n$? Prove that your answer is a lower bound.\\\\
\textit{Note: On every problem set we will provide a challenge problem, generally significantly more difficult than the other problems in the set, but worth only a few points. It is recommended that you attempt these problems, but only after completing the rest of the assignment.} 
\\\\
\noindent {\bf Solution.}
\\\noindent 
{(A) }  
\\\noindent 
It is injective. Each digit is 0 or 1. If the $i^{th}$  character is 1, you add the $i$ to the set. So different $n$-digit numbers are transformed to different elements of $P(S_n)$ by $f_n$ because each digit is $0$ or $1$. 
\\\\\noindent 
It is surjective. Each digit is 0 or 1.  Elements of $P(S_n)$ represent the $i^{th}$ digits that are 1 in $n$-digit number. Power set of $S_n$ is all the digits that are 1 in all the possible $n$-digit numbers. Since different numbers have different locations of 0 and 1, $P(S_n)$ means the all the different possible $n$-digit numbers that made of 0 and 1. Elements of $P(S_n)$ are hit at least by once. Thus, this is surjective.
\\\\\noindent 
Since it is both surjective and injective, this is bijective.
\\\\\noindent 
(C)
We need $ 2^n $ elements of $P(S_n)$ to explain things. Since  $P(S_n)$ has $ 2^n $  elements and the formula is bijective, 
we need at least $ 2^n $ states of DFA to compute $f_n$. 

\pagebreak
%%%%%%%%%%%%%%%%%%%%%%%%%%%%%%%%%%%%%%%%%%%%%%%
\PART{Juan and Varun}
%%%%%%%%%%%%%%%%%%%%%%%%%%%%%%%%%%%%%%%%%%%%%%%
\problem{5+5}{1/2 page}

Two FAs are ``equivalent'' if the languages that they accept are the same. If two FAs are not equivalent, they are ``distinct''. For this question, you may assume that the alphabet is $\{0, 1\}$.

\subproblem How many distinct DFAs are there with 1 state? Draw it/them. Describe informally the languages recognized by each.

\subproblem How many distinct NFAs are there with 1 state? Draw it/them. Describe informally the languages recognized by each.
\\\\
\noindent {\bf Solution.}
\\\noindent Since DFAs with one state can only have a transition to itself, we only have two distinct DFAs. 
\\
\\\noindent Since NFAs with one state can have a transition to nowhere, we have five distinct NFAs. 

\problem{3+3+3}{1/3 page}
Are the following statements true or false? Justify your answers with a proof or counterexample.\\
\subproblem $(L_1 \cap L_2)^* = L_1^* \cap L_2^*$

\subproblem $(L_1 \cup L_2) \cdot L_3 = (L_1 \cdot L_3) \cup (L_2 \cdot L_3)$, where $\cdot$ is concatenation.

\subproblem $\{ \vareps\} \cdot L_1 = \emptyset \cdot L_1$
\\\\
\noindent {\bf Solution.}


\noindent {(A) }  
\\ if $L_1$ has $\{\ aa \} $ and $L_2$ has $\{\ a \} $, $ L_1 \cap L_2 =\emptyset  $. Thus, $(L_1 \cap L_2)^* =\{\vareps \} $.
\noindent However, $L_1^* \cap L_2^* =\{\ aa \} ^*$


\noindent {(B) }
$(L_1 \cup L_2) = \set{z : z\in L_1 ,z\in L_2 }$
\\ Thus, $(L_1 \cup L_2) \cdot L_3 = \set{zx : z\in L_1 ,z\in L_2 ,x\in L_3 } $
\\$(L_1 \cdot L_3) =\set{zx : z\in L_1 ,x\in L_3 } $
\\$(L_2 \cdot L_3) =\set{yx : y\in L_2 ,x\in L_3 } $
\\$(L_1 \cdot L_3) \cup (L_2 \cdot L_3)=\set{yx : y\in L_2 ,x\in L_3 }  \cup \set{zx : z\in L_1 ,x\in L_3 } =\set{ax : a\in L_1 ,a\in L_2 ,x\in L_3 } $
This is the same as $\set{zx : z\in L_1 ,x\in L_3 } $.
Thus $(L_1 \cup L_2) \cdot L_3 = (L_1 \cdot L_3) \cup (L_2 \cdot L_3)$ is true.

\noindent {(C) }  
\\ $\{ \vareps\} \cdot L_1 =\set{yx : y\in \{\vareps\} ,x\in L_1 } =L_1 $  
\\ Concatinating an empty string to y becomes y.
\\ $ \emptyset \cdot L_1 =\set{yx : y\in {\emptyset} ,x\in L_1 } =\emptyset $
\\ Since there is no y, this makes $\emptyset $

\noindent Thus they are not the same.



\end{document}